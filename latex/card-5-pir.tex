\documentclass{pocketcard}

\begin{document}

\centering
\textbf{\Large DECK 5: POST-INCIDENT REVIEW}
\vspace{0.1cm}
\hrule
\textbf{\footnotesize BLAMELESS CULTURE, LEARNING, \& WORK-AS-DONE}

\vspace{0.5cm}

\cardsection{1. Blameless Culture Mandate}

\subsectiontitle{THE GOLDEN QUESTION}
\begin{itemize}
    \item \textbf{Shift Focus}: The person who triggered the failure is a resource for learning, not the root cause.
    \item \textbf{No Single Point}: Assume failure was caused by flaws in tools, systems, or process design, not personal failings.
    \item \textbf{Ask "Why 5x"}: Stop the investigation when the answer points to a lack of investment or systemic pressure, not human error.
\end{itemize}

\cardsection{2. The STELLA Framework}

\subsectiontitle{COMPARING WORK-AS-IMAGINED VS. WORK-AS-DONE}
\begin{tabularx}{\linewidth}{|p{2.3cm}|X|}
\hline
\textbf{Script vs. Reality} & What was the written procedure vs. what did the operator *actually* do? \\
\hline
\textbf{Local Rationality} & What made the operator's actions make sense to \textbf{them} at the time? \\
\hline
\textbf{Pressures} & Which incentives (time, manager, customer) pushed the operator to deviate? \\
\hline
\textbf{Success Factors} & List three things the team did right that prevented a worse outcome. \\
\hline
\end{tabularx}

\cardsection{3. Action Item Mandates}

\subsectiontitle{LEARNING FROM THE EVENT}
\begin{itemize}
    \item \textbf{Tooling over Training}: If a mistake happened, fix the tool or the process, not the person.
    \item \textbf{Risk Reduction}: Actions must focus on reducing the likelihood of high-consequence events, not fixing low-impact bugs.
    \item \textbf{Owners \& Dates}: Every action item must have a single owner and an estimated completion date.
\end{itemize}

\end{document}
